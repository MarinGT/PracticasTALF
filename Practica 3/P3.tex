\documentclass{article}
\usepackage[utf8]{inputenc}
\usepackage{graphicx}
\usepackage[a4paper,top=3cm,bottom=2cm,left=3cm,right=3cm,marginparwidth=2cm]{geometry}

\title{Teoría de Autómatas y Lenguajes Formales\\[.4\baselineskip]Práctica 3}
\author{Marina González Torres}
\date{}

\begin{document}

\maketitle

\section*{Ejercicio 1}
\subsection*{Defina la solución de la Maquina de Turing del ejercicio 3.4 de la lista de ejercicio y pruebe su correcto funcionamiento.}
El ejercicio 3.4 nos dice que debemos probar que la funcion add(x, y) con x e y perteneciente a los números naturales es computable por Turing usando la notación unaria \{\}
\\
\\ La máquina de Turing simulada con el JFlap quedaría de la siguiente forma:
\\ \includegraphics[width=18.5cm]{Ejercicio 1.jpg}
Para poder visualizar el correcto funcionamiento adjunto un ejemplo 
 de ejecución múltiple:
\\
\\ \includegraphics[width=15.5cm]{Demostracion.png}

\section*{Ejercicio 2}
\subsection*{Defina una función recursiva para la suma de tres valores.}
Para realizar este ejercicio hemos usado unos ficheros existentes en el repositorio de "talfuma" llamados "recursivefunctions" y "evalrecfunction".
\\ En el primer fichero hemos añadido una funcion a la que hemos llamado addition2 que realiza la función recursiva de tres valores.
\\
\\ \includegraphics[width=10cm]{FuncionEj2.png}
\\
\\ Luego hemos probado su funcionamiento con Octave y un ejemplo de ello sería el siguiente:
\\
\\ \includegraphics[width=10cm]{OctaveEj2.png}
\\

\section*{Ejercicio 3}
\subsection*{Implemente un programa WHILE que calcule la suma de tres valores. Debe utilizar una variable auxiliar que acumule el resultado de la suma.}
Para realizar este ejercicio hemos usado los recursos proporcionados en el fichero "F.m" en talfuma.
\\La función WHILE que implementa la suma de los tres valores quedaría de la siguiente manera:
\\
\\ \includegraphics[width=10cm]{Ejercicio3.png}
\\
\\ Ejemplo de ejecución:
\\
\\ \includegraphics[width=15cm]{Ejercicio3ej.png}

\end{document}
