\documentclass{article}
\usepackage[utf8]{inputenc}
\usepackage{graphicx}
\usepackage[a4paper,top=3cm,bottom=2cm,left=3cm,right=3cm,marginparwidth=2cm]{geometry}

\title{Teoría de Autómatas y Lenguajes Formales\\[.4\baselineskip]Práctica 3}
\author{Marina González Torres}
\date{}

\begin{document}

\maketitle

\section*{Ejercicio 1}
\subsection*{Cree el programa WHILE más simple que calcule la función de divergencia (con cero argumentos) y calcule la codificación de su código.}
Para realizar este ejercicio vamos a usar el fichero proporcionado llamado "CODE2N". Dicho fichero nos va a devolver la codificación del código WHILE que le pasemos como parametro.
\\
\\ Para que el código diverja tenemos que realizar un bucle infinito.
\\
\\ \includegraphics[width=8cm]{EJ1P4.png}
\\
\\ Sin embargo, al ejecutarlo con Octave se produce un error sin saber muy bien el motivo:
\\
\\ \includegraphics[width=14cm]{errorEJ1P4.png}

\newpage
\section*{Ejercicio 2}
\subsection*{Cree un script de Octave que enumere todos los vectores.}
Para crear el scrip también usaremos el script dado llamado "godeldecoding". El script será el siguiente:
\begin{verbatim}
    function MostrarVectores(N)
    for i=0: N-1
        fprintf('Vector %s: (%s)\n', num2str(i), num2str(godeldecoding(i)))
    end
    end
\end{verbatim}
Un ejemplo de su ejecución con Octave sería el siguiente:
\\
\\ \includegraphics[width=7cm]{EJ2P4.png}

\section*{Ejercicio 3}
\subsection*{Cree un script de Octave que enumere todos los programas WHILE.}
Para crear este script usaremos "N2WHILE" que es un script ya proporcionado. El script quedaría de la siguiente forma:
\begin{verbatim}
    function MostrarWHILE(N)
    for i=0: N-1
        fprintf('Programa %s: %s\n', num2str(i), N2WHILE(i))
    end
    end
\end{verbatim}
Un ejemplo de su ejecución con Octave sería el siguiente:
\\
\\ \includegraphics[width=7cm]{EJ3P4.png}
\\
\\El igual del WHILE se ve desplazado aunque desconozco el motivo.

\end{document}