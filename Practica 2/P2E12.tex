\documentclass{article}
\usepackage[utf8]{inputenc}
\usepackage[spanish]{babel}
\usepackage{amsthm, amsmath}
\usepackage{enumitem}
\usepackage{graphicx}
\usepackage[a4paper,top=3cm,bottom=2cm,left=3cm,right=3cm,marginparwidth=2cm]{geometry}

\title{Practica 2}
\author{Marina Gonzalez Torres}
\date{}

\begin{document}

\maketitle
\setlength{\parindent}{0pt}

\section*{Ejercicios}

\subsection*{Ejercicio 1}
Consideremos el lenguaje sobre el alfabeto \{a,b\} que sólo contiene a la cadena a.

\begin{enumerate}[label=\alph{enumi})]
    \item Construye un AFD que reconozca este lenguaje y que rechace todas aquellas cadenas que no pertenezcan al lenguaje.
    \item Prueba el autómata que has creado mediante la introducción de 6 cadenas.
\end{enumerate}

\underline{\textbf{Solución:}}
\\
\\ \underline{Construcción del lenguaje:}
\begin{equation*}
    K = \{q_0, q_1, q_2\}
\end{equation*}
\begin{equation*}
    \Sigma = \{a, b\}
\end{equation*}
\begin{equation*}
    s = q_0
\end{equation*}
\begin{equation*}
    F = \{q_1\}
\end{equation*}
\begin{equation*}
    \bigtriangleup = \{(q_0,a,q_1), (q_0,b,q_2), (q_1,a,q_1), (q_1,b,q_2), (q_2,a,q_2), (q_2,b,q_2)\}
\end{equation*}

\\ \underline{La prueba en JFLAP quedaría de la siguiente forma:}
\\ \includegraphics[]{EJ1.png}

\subsection*{Ejercicio 2}
Automata finito en Octave:

\begin{enumerate}[label=\alph{enumi})]
    \item Abre el script de Octave \textit{finiteautomata.m} y pruebalo con el ejemplo dado en el repositorio GitHub.
    \item Especifica en \textit{finiteautomata.json} el autómata creado en la actividad 1 y pruebalo con el script.
\end{enumerate}

\underline{\textbf{Solución:}}
\\
\\ \underline{Probar el automata de GitHub:}
\\
\\ \includegraphics[width=16cm]{EJ2a.png}
\newpage
\underline{Especificar en el json el automata del ejercicio 1:}
\begin{verbatim}
    {
        "name" : "aa*",
        "representation" : {
            "K" : ["q0", "q1", "q2"],
            "A" : ["a", "b"],
            "s" : "q0",
            "F" : ["q1"],
            "t" : [["q0", "a", "q1"],
                   ["q0", "b", "q2"],
                   ["q1", "a", "q1"],
                   ["q1", "b", "q2"],
                   ["q2", "a", "q2"],
                   ["q2", "b", "q2"]]
        }
    }
\end{verbatim}

\textit{Prueba:}
\\
\\ \includegraphics[width=16cm]{EJ2b2.png}

\end{document}