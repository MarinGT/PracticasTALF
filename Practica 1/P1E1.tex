\documentclass{article}
\usepackage[utf8]{inputenc}
\usepackage[spanish]{babel}
\usepackage{amsthm, amsmath}
\usepackage[a4paper,top=3cm,bottom=2cm,left=3cm,right=3cm,marginparwidth=2cm]{geometry}

\title{Teoría de Autómatas y Lenguajes Formales\\[.4\baselineskip]Práctica 1 - Ejercicio 1}
\author{Marina González Torres}
\date{}

\begin{document}

\maketitle
\setlength{\parindent}{0pt}

\section*{Enunciado}
Encuentre el conjunto potencia $\mathcal R^{3}$ de la relación $\mathcal R = \{(1,1), (1,2), (2,3), (3,4)\}$.
\\Comprueba tu respuesta con el scrip \textit{powerrelation.m} y escribe en un documento en \LaTeX \ con la solucion paso a paso.

\section*{Potencia de una relacion $\mathcal R^{n}$}

\begin{equation*}
    R^{n} =
    \begin{cases}
        R & n = 1 
        \\ \{(a,b) : \exists x \in A, (a,x) \in R^{n-1} \wedge (x,b) \in R\} & n > 1
    \end{cases}
\end{equation*}

\section*{Pasos:}
\section{Hacemos $\mathcal R^{2}$}
Para hacer $\mathcal R^{2}$ usamos la regla de la potencia de una relacion cuando n $\mathcal > 1$.

\begin{center}
    $(1,1) : (1,1) \in \mathcal R^{1} \wedge (1,1) \in \mathcal R$
    \\$(1,2) : (1,1) \in \mathcal R^{1} \wedge (1,2) \in \mathcal R$
    \\$(1,3) : (1,2) \in \mathcal R^{1} \wedge (2,3) \in \mathcal R$
    \\$(2,4) : (2,3) \in \mathcal R^{1} \wedge (3,4) \in \mathcal R$
\end{center}
Por tanto tendriamos que $\mathcal R^{2} = \{(1,1), (1,2), (1,3), (2,4)\}$

\section{Hacemos $\mathcal R^{3}$}
Para hacer $\mathcal R^{2}$ usamos la regla de la potencia de una relacion cuando n $\mathcal > 1$.

\begin{center}
    $(1,1) : (1,1) \in \mathcal R^{2} \wedge (1,1) \in \mathcal R$
    \\$(1,2) : (1,1) \in \mathcal R^{2} \wedge (1,2) \in \mathcal R$
    \\$(1,3) : (1,2) \in \mathcal R^{2} \wedge (2,3) \in \mathcal R$
    \\$(1,4) : (1,3) \in \mathcal R^{2} \wedge (3,4) \in \mathcal R$
\end{center}
Por tanto tendriamos que $\mathcal R^{3} = \{(1,1), (1,2), (1,3), (1,4)\}$

\section*{Solución: $\mathcal R^{3} = \{(1,1), (1,2), (1,3), (1,4)\}$}

\end{document}