\documentclass{article}
\usepackage[utf8]{inputenc}
\usepackage[spanish]{babel}
\usepackage{amsthm, amsmath}
\usepackage{nccmath}
\usepackage{graphicx}
\usepackage{enumitem}
\usepackage[a4paper,top=3cm,bottom=2cm,left=3cm,right=3cm,marginparwidth=2cm]{geometry}

\title{Teoría de Autómatas y Lenguajes Formales\\[.4\baselineskip]Práctica 1 - Ejercicio 2}
\author{Marina González Torres}
\date{}

\begin{document}

\maketitle
\setlength{\parindent}{0pt}

\section*{Enunciado}
Dentro de la carpeta 'files', encontramos un archivo TEX en cuyo contenido aparece la cadena $\backslash$usepackage\{amsthm, amsmath\}. Nota: usa grep e ignora los caracteres especiales con $\backslash$. Completa la demostración y responde la pregunta.

\subsection*{Comando usado}
Para encontrar en la carpeta 'files' el archivo usamos el comando:
\\ \textbf{grep "$\backslash$usepackage\{amsthm, amsmath\}" \ files /*}
\\ Tras usar este comando obtenemos que el archivo que contiene dicha cadena es mainP.tex
\\ 
\\A continuación muestro lo obtenido con la consola:
\\
\\ \includegraphics[width=14cm, height=4cm]{ConsolaP1E2.png}

\clearpage
\subsection*{mainP.tex}
El archivo contiene el codigo en \LaTeX \ que genera el siguiente documento:
\\
\\ \includegraphics{mainP.png}

\clearpage
\section*{Resolucion}
\underline{Nos piden completar la demostración:}
\\
\newcommand{\Lb}{\pazocal{L}}

Si $\alpha,\beta,\gamma$ son expresiones regulares entonces se cumple:
\begin{equation}
    (\alpha+\beta)\gamma=\alpha\gamma+\beta\gamma
\end{equation}


\begin{proof*}
Usando la definición descrita en el mainP.tex tenemos que:
    \begin{multline*}
        \Lb(((\alpha+\beta)\gamma))= \Lb((\alpha+\beta))\Lb(\gamma)= (\Lb(\alpha)\cup \Lb(\beta))\Lb(\gamma)= \\ \Lb(\alpha)\Lb(\gamma)\cup \Lb(\beta)\Lb(\gamma)= \Lb(\alpha\gamma) \cup \Lb(\beta\gamma) = \Lb(\alpha\gamma + \beta\gamma)
    \end{multline*}
\end{proof*}

\underline{Además, nos piden completar el ejemplo:}

\newtheorem{example}{Ejemplo}
\begin{example}
Consideremos $L=\{w\in \{a,b\}^* : w \textnormal{ no termina en } ab\}$. Un expresión regular que genera L N es: \\
(\epsilon+a)(a+b)*a

\end{example}

\end{document}